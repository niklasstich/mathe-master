\documentclass{article}
\usepackage{mathtools}
\usepackage{amsfonts}
\usepackage{enumerate}
\usepackage[utf8]{inputenc}
\usepackage[ngerman]{babel}

\newtheorem{theorem}{Satz}[section]
\newtheorem{subtheorem}{Satz}[theorem]

\title{Mathematik (Master) für Informatik Skript}
\date{SoSe 2021}
% Bei PR gerne Namen hinzufügen, wenn du willst
\author{erstellt von Prof. Dr. Preisenberger\\ transkribiert von Niklas Stich, Alexander Walk}

\begin{document}
\begin{titlepage}
\maketitle
\tableofcontents
\end{titlepage}


\section{Komplexe Zahlen und komplexwertige Funktionen}
\subsection{Historisches}
16. Jahrhundert: Lösung algebraischer Gleichungen\\
Gegeben: \(a_{k}\in\mathbb{Q},\quad k=0,\dots,n\in\mathbb{N}_{0}\)
\begin{theorem}
\begin{align}
    a_{n}x^{n}+a_{n-1}x^{n-1}+\dots+a_{1}x+a_{0} &= 0, \quad a_{n}\neq 0 \\
    \sum_{k=0}^{n}a_{k}^{k} &= 0 \\
    \underbrace{p(x)}_\text{Polynom n-ten Grades}&=0, \quad p \in \mathbb{Q}[x]
\end{align}
\end{theorem}
Gesucht: Lösung von (1.1) zunächst für $n \leq 3$\\
bekannt war:
\begin{enumerate}[(i)]
    \item $n=1,\quad a_{1}x+a_{0}=0,\quad x=-\dfrac{a_{0}}{a_{1}}$
    \item $\begin{aligned}[t]
        n&=2: & a_{2}&x^{2}+a_{1}x+a_{0} & &=0\\
        a_{2}&\neq 0, & &x^{2}+\dfrac{a_{1}}{a_{2}}x+\dfrac{a_{0}}{a_{2}} & &= 0\\
        & & &x^{2}+2\dfrac{a_{1}}{2a_{2}}x & &= - \dfrac{a_{0}}{a_{2}}\\
        & & &x^{2}+2\dfrac{a_{1}}{2a_{2}}x +(\dfrac{a_{1}}{2a_{2}})^2 & &= (\dfrac{a_{1}}{2a_{2}})^2 - \dfrac{a_{0}}{a_{2}}\\
        & & (&x+\dfrac{a_{1}}{2a_{2}})^2 & &= (\dfrac{a_{1}}{2a_{2}})^2 - \dfrac{a_{0}}{a_{2}}\\
        & & &x+\dfrac{a_{1}}{2a_{2}} & &=\pm\sqrt{\dfrac{a_{1}^{2}}{4a_{2}^{2}} - \dfrac{4a_{2}a_{0}}{4a_{2}^{2}}}\\
        & & &x_{1/2} & &=-\dfrac{a_{1}}{2a_{2}}\pm\dfrac{\sqrt{a_{1}^{2}-4a_{0}a_{2}}}{2a_{2}}
        \end{aligned}$\\
		Falls $\begin{aligned}[t]
            &\Delta = a_{1}^{2}-4a_{0}a_{2} & &< 0 & &\Rightarrow \text{keine reelle Lösung}\\
            &\Delta & &=0 & &\Rightarrow \text{genau eine Lösung}\\
            &\Delta & &> 0 & &\Rightarrow \text{genau zwei Lösungen}
        \end{aligned}$\\
        Die Lösungen sind die Schnittpunkte einer Parabel mit der x-Achse. Die Schnittpunkte existieren für Parabeln mit $\Delta \geq 0$.
        Aber für ${a_{2}=1,\,a_{0},\,a_{0}=-2}$ ergibt sich: $x_{1/2} = \pm \sqrt{2}\notin \mathbb{Q}$.\\
        Folge: Erweiterung von $\mathbb{Q}$ zu $\mathbb{Q}(\sqrt{2})=\{a + b\sqrt{2}\,|\,a,b\in\mathbb{Q}\}$.
        $\mathbb{Q}(\sqrt{2})$ ist ein Körper, in dem wie in $\mathbb{Q}$ gerechnet werden kann und in dem die Lösungen von $x^{2}=2$ existieren.
    \item Für Gleichungen 3. Grades keine allgemeine Lösungsformel, jedoch für spezielle Gleichungen\\\\
        Cardano: \\
        Für bestimmte Gleichungen 3. Grades können alle ihre reellen Lösungen mit Lösungen von $x^{3}=\alpha$ und $x^{2}=\beta=-1$ beschrieben werden,
        sofern man mit diesen Lösungen wie in $\mathbb{R}$ rechnet.\\\\
        Ergebnis: Einführung einer neuen Zahl i mit $i^{2} = -1$
        Damit waren die imaginären bzw. komplexen Zahlen erfunden.\\
        $\rightarrow$ Blüte des 19. Jahrhunderts 
    \item Galois, Abel (19. Jht.)\\
        Für die Lösungen der Gleichung $\sum_{k=0}^{n}a_{k}x^{k} = 0$ gibt es nur für $n\leq4$ allgemeingültige,
        von den Koeffizienten $a_{k}$ abhängige, Lösungsformeln.\\
        Für $n\geq5$ gibt es $\underline{keine}$ allgemeingültigen Formeln.
\end{enumerate}
\subsection{Der Körper der komplexen Zahlen \quad \; (Hartmann 5.3)}
    Wir definieren: $\mathbb{C} = \{ (a,b) | a,b \in \mathbb{Q}\}$\\
    Gleichheit: $(a,b),(c,d) \in \mathbb{C}, (a,b)=(c,d) \Leftrightarrow a=c, b=d\\
    "\!+\!":\mathbb{C}\times\mathbb{C}\rightarrow\mathbb{C},((a,b),(c,d))\mapsto ((a,b)+(c,d))=(a+c,b+d)\\
    "\ast":\mathbb{C}\times\mathbb{C}\rightarrow\mathbb{C},((a,b),(c,d))\mapsto ((a,b)\ast(c,d))=(a\ast c-b\ast d,a\ast d+b\ast c) $
\setcounter{theorem}{2}
\begin{subtheorem}
    Es gilt $(\mathbb{C},"+","\ast")$ ist ein Körper, d.h.
    \begin{enumerate}
        \item $(\mathbb{C},"+")$ ist eine abelsche Gruppe mit dem neutralen Element $(0,0) = 0_{\mathbb{C}}$
        \item $(\mathbb{C}\setminus \{0_{\mathbb{C}}\},"\ast")$ ist eine abelsche Gruppe mit dem neutralen Element\\
         $(1,0)=1_{\mathbb{C}}$;\, inverses Element: weiter unten.
         \item Distributivgesetz: $\forall x,y,z\in\mathbb{C}:(x+y)\ast z=x\ast z+y\ast z$\\
         Beweis: Nachrechnen als Übung.

    \end{enumerate}
\end{subtheorem}
\end{document}
